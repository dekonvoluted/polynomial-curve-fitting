
\documentclass{article}

\usepackage[utf8]{inputenc} % Encoding on input fonts set to UTF-8
\usepackage[T1]{fontenc} % Font encoding set to T1 fonts
\usepackage{fullpage} % Margins reduced to 1 in
\usepackage{mathptmx} % Use Times font
\usepackage[version=3]{mhchem} % Use chemical formulae
\usepackage[colorlinks]{hyperref} % Use links

\title{Polynomial Curve Fitting}
\author{Karthik Periagaram}
\date{\today}

\begin{document}

\maketitle

\section{A Note on Notation}

In what follows, scalars, (column-)vectors and matrices are represented consistently.
A scalar, such as \(n\) (the number of observations) is written in lowercase.
A vector, such as \(\vec{X}\) (the input vector) is always a column vector and is written in uppercase and has an arrow over it.
Finally, a matrix is written in uppercase, example, \(A\vec{X}=\vec{B}\).

\section{Problem Statement}

\section{Solution}


\bibliographystyle{plain}
\bibliography{}

\end{document}

